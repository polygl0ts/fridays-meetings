\documentclass[aspectratio=169]{beamer}

\usepackage{../theme/polygl0ts}
\usepackage{drawstack}

\tikzstyle{freecell}=[fill=none]

\newcommand{\reg}[1]{\%\mintinline{asm}{#1}}
\newcommand{\hex}[1]{\mintinline{python}{0x#1}}
\newcommand{\naddr}[2]{\begin{tabular}{l}#1\\\hex{#2}\end{tabular}}

\setbeamertemplate{navigation symbols}{}

%%%%%%%%%%%%%%%%%%%%%%%%%%%%%%%%%%%%%%%%%%%%%%%%%%%%%%%%%%%%%%%%%%%%%%%%%%%%%%%
% Title Setup
%%%%%%%%%%%%%%%%%%%%%%%%%%%%%%%%%%%%%%%%%%%%%%%%%%%%%%%%%%%%%%%%%%%%%%%%%%%%%%%
\title{Lesson 1: Buffer Overflows}
\subtitle{Getting your feet wet, by pwning your first binary.}
\author{Leonardo Galli}
\date{\today}

\begin{document}
\titleframe

\tocframe

\section{Introduction}

\begin{frame}
    \frametitle{Information}
    \begin{itemize}
        \item Slides and additional material on our website \href{https://flagbot.ch}{flagbot.ch} (under materials)
        \item Read the slides for Lesson 0, if you are new
        \item Subscribe to the mailinglist for receiving information: \href{https://lists.vis.ethz.ch/postorius/lists/ctf.lists.vis.ethz.ch/}{lists.vis.ethz.ch} (CTF, not CTF-announce)
    \end{itemize}
    \begin{center}
        \includegraphics[width=3cm]{mailing-list.png}
    \end{center}
\end{frame}

\begin{frame}
    \frametitle{Disclaimer}
    \begin{alertblock}{\textbf{Ethical Hacking}}
    {
        In these lessons you will gain firsthand experience with methods used to exploit all kinds of systems.
        Our purpose is mostly to help you get better at solving CTF challenges, so we strongly urge you to only practice ethical hacking.\\
        We do not condone trying to gain access to any system you are not specifically authorized to do so.
        If you do find any vulnerabilities in software, always report it through the proper channels!
    }
    \end{alertblock}    
\end{frame}

\begin{frame}
    \frametitle{Disclaimer}
    {
        \setbeamercolor{block title alerted} {fg = white, bg=brightgreen}
        \setbeamercolor{block body alerted} {bg = darkgreen}
    \begin{alertblock}{\textbf{Ethical Hacking}}
    {
        In these lessons you will gain firsthand experience with methods used to exploit all kinds of systems.
        Our purpose is mostly to help you get better at solving CTF challenges, so we strongly urge you to only practice ethical hacking.\\
        We do not condone trying to gain access to any system you are not specifically authorized to do so.
        If you do find any vulnerabilities in software, always report it through the proper channels!
    }
    \end{alertblock}
    }
\end{frame}

\section{Setting up the Environment}
\begin{frame}[fragile]
    \frametitle{How to not Infect Yourself}
    \begin{itemize}
        \item Most of the time, challenges are geared towards Linux
        \item Thus, you will probably need to setup a virtual machine
        \item Even if you run linux natively, setting up a vritual machine has a lot of benefits:
        \begin{itemize}
            \item No risk when running random binaries on your computer
            \item Tooling is setup and ready to go immediately
            \item If you fuck something up, just run \inlinecode[bash]{vagrant destroy && vagrant up} and you are good to go
            \item Different VMs for different libc versions
        \end{itemize}
    \end{itemize}
    \pause
    \begin{alertblock}{\textbf{Kali et al.}}
        {
            We strongly advise against using distros built for "hacking".
            While the tools they provide by default can be nice in some scenarios, CTFs often require quite different tools and challenges often work best on standard distributions such as ubuntu.
        }
    \end{alertblock}
\end{frame}

\begin{frame}
    \frametitle{Making your Life Easy}
    \begin{itemize}
        \item We have prepared tooling for you to setup a VM that is geared towards CTFs
        \item For detailed config instructions, read the config.rb file found once you download the repo
        \item Start downloading and setting everything up now, so that you can participate in the challenge later :)
    \end{itemize} 
\end{frame}

\begin{frame}[fragile]
    \frametitle{Instructions For Now}
     \begin{enumerate}
        \item Download and Install VirtualBox (\href{https://www.virtualbox.org/wiki/Download_Old_Builds_6_0}{Click on your OS here})
        \item Download and Install Vagrant (\href{https://www.vagrantup.com/downloads.html}{here})
        \item Clone our git repository from \href{https://gitlab.ethz.ch/vis/ctf/ctf-vm}{here}: \inlinecode[bash]{git clone https://gitlab.ethz.ch/vis/ctf/ctf-vm.git}
        \item Change config.rb to your liking (most likely you only need to change your shared folders and the ssh key)
        \item Run \inlinecode[bash]{vagrant up 27} to start the download and installation of the VM.
    \end{enumerate}
            \begin{codebox}{ruby} 
:shared_folders => {
    "/path/to/my/projects" => "/home/vagrant/CTF"
},
# ...
:ssh_key => "/path/to/my/user/.ssh/id_rsa.pub",
\end{codebox}
\end{frame}

\section{Buffer Overflow}
\subsection{The Stack}
\begin{frame}[fragile]
    \frametitle{The Stack}
    \begin{itemize}
        \item Before we can start overflowing buffers, we need to learn about "the stack" (AKA the call stack)
        \item stack data structure, that stores information about the active functions (or subroutines) of a computer program
        \item Every function has a so called "stack frame" - on the stack - where information is stored, such as
        \begin{itemize}
            \item local variables (numbers, strings, arrays)
            \item saved instruction pointer - the return address (used to know where to return back to, when the function is done)
            \item additional arguments (not really relevant yet)
        \end{itemize}
        \item Local variables allocated in the stack go from low to high addresses, i.e. we may find the first character of \inlinecode[c]{"Hello World"} at \hex{7fa0}, while the last character would be at \hex{7fab}
        \item However, the stack grows downwards, i.e. if we push items a, b on the stack (in that order), we may find a at \hex{7698}, while b would be at \hex{7690}
    \end{itemize}

\end{frame}
\begin{frame}[fragile]
    \frametitle{Example}
    \begin{columns}
    \begin{column}{0.5\textwidth}
        \begin{codebox}{c}
int callme() {
    char name[16];
    gets(name); |\nline{3}|
    printf("Hello %s", name); |\nline{4}|
    return 2; |\nline{5}|
}

int main(int argc, char* argv[]) {
    long x = 4; |\nline{1}|
    long y = callme(); |\nline{2}|
    printf("x + y = %d", x + y); |\nline{6}|
    return 0; |\nline{7}|
}
        \end{codebox}
    \end{column}
    \begin{column}{0.5\textwidth}
        \begin{figure}[H]
            \centering
            \begin{drawstack}[scale=0.7, every node/.style={scale=0.7}]
                \startframe
            \cell{\only<1-5>{\hex{ZZZZZZZZ}}\only<6->{\mintinline{c}{2}}} \cellcom{\naddr{y}{7fd0}}
                \cell{\only<1>{\hex{ZZZZZZZZ}}\only<2->{\mintinline{c}{4}}} \cellcom{\naddr{x}{7fc8}} 
                \finishframe{{\tt main}\\stack frame}
                \startframe
                \cell{\only<1-2>{\hex{ZZZZZZZZ}}\only<3->{\hex{7fe0}}} \cellcom{\naddr{saved \reg{rbp}}{7fc0}}
                \cell{\only<1-2>{\hex{ZZZZZZZZ}}\only<3->{\hex{4012ae}}} \cellcom{\naddr{saved \reg{rip}}{7fb8}}
                \cell{} \cellcom{\naddr{\mintinline{c}{name[8:15]}}{7fb0}}
                \node[draw,minimum height=1cm,anchor=east] (n8) at (2, \value{cellnb}) {\only<1-3>{\mintinline{c}{n[8]}}\only<4->{\mintinline{c}{'\0'}}};
                \node[] at (0, \value{cellnb}) {\dots};
                \node[draw,minimum height=1cm,anchor=west] (n15) at (-2, \value{cellnb}) {\only<1-3>{\mintinline{c}{n[15]}}\only<4->{\mintinline{c}{'\0'}}};
                \cell{} \cellcom{\naddr{\mintinline{c}{name[0:7]}}{7fa8}}
                \node[draw,minimum height=1cm,anchor=east] (n0) at (2, \value{cellnb}) {\only<1-3>{\mintinline{c}{n[0]}}\only<4->{\mintinline{c}{'L'}}};
                \node[] at (0, \value{cellnb}) {\dots};
                \node[draw,minimum height=1cm,anchor=west] (n7) at (-2, \value{cellnb}) {\only<1-3>{\mintinline{c}{n[7]}}\only<4->{\mintinline{c}{'o'}}};
                \finishframe{{\tt callme}\\stack frame}

                \draw[->] (2.15,0) -- node[auto, sloped]{stack growth} (2.15, \value{cellnb}-1);
            \end{drawstack}
              
            
            \caption{The Stack}
            \label{fig:stack}
        \end{figure}
    \end{column} 
    \end{columns}

\end{frame}
\subsection{Overflow}
\begin{frame}
    \frametitle{C is hard}
    \begin{itemize}
        \item C is a very low level language
        \item Basically no checks in place!
        \begin{itemize}
            \item No type checks, no checks on array sizes, etc.!
        \end{itemize}
        \item At every step the programmer needs to make sure, user input is correct
    \end{itemize}
\end{frame}
\begin{frame}[fragile]
    \frametitle{Programming Mistakes}
    \begin{itemize}
        \item Common mistakes are using functions that do not check sizes of strings, arrays, etc.
        \begin{itemize}
            \item Examples are: \inlinecode[c]{gets, strcpy, sprintf, strcat, ...}
        \end{itemize}
        \item If you see them used in a program, immediately assume there is an attack vector here
        \item How can we exploit this?
    \end{itemize}
    \begin{alertblock}{Exploit?}
        Almost always, the "holy grail" is getting code execution when trying to exploit the attack vector.
        Because these programs are either run on a server or - more general - a remote computer, this allows us to execute code on it.\\
        However, often this might entail using multiple different attack vectors.
        Furthermore, sometimes reading arbitrary memory can already be enough.
    \end{alertblock}
\end{frame}
\subsection{ROP}
\begin{frame}
    \frametitle{Getting Code Execution}
    \begin{itemize}
        \item If the programmer fails to check the length of our input, we can overwrite stuff on the stack
        \item What of interest is stored on the stack?
        \pause
        \item If we overwrite the saved instruction pointer, we can control where the function returns to
        \item We have effectively gained code execution!
        \item Let's give the example program 32 characters of A (\hex{41} in hex)
    \end{itemize}

\end{frame}
\begin{frame}[fragile]
    \frametitle{Example 2.0: Smashing the Stack for Fun and Profit}
    \begin{columns}
        \begin{column}{0.5\textwidth}
            \begin{codebox}{c}
int callme() {
    char name[16];
    gets(name); |\nline{1}|
    printf("Hello %s", name); |\nline{2}|
    return 2; |\nline{3}|
}

int main(int argc, char* argv[]) {
    long x = 4;
    long y = callme();
    printf("x + y = %d", x + y);
    return 0;
}
            \end{codebox}
        \end{column}
        \begin{column}{0.5\textwidth}
            \begin{figure}[H]
                \centering
                \begin{drawstack}[scale=0.7, every node/.style={scale=0.7}]
                    \startframe
                \cell{\mintinline{c}{2}} \cellcom{\naddr{y}{7fd0}}
                    \cell{\hex{ZZZZZZZZ}} \cellcom{\naddr{x}{7fc8}} 
                    \finishframe{{\tt main}\\stack frame}
                    \startframe
                    \cell{\only<1>{\hex{7fe0}}\only<2->{\hex{4141414141414141}}} \cellcom{\naddr{saved \reg{rbp}}{7fc0}}
                    \cell{\only<1>{\hex{4012ae}}\only<2->{\hex{4141414141414141}}} \cellcom{\naddr{saved \reg{rip}}{7fb8}}
                    \cell{} \cellcom{\naddr{\mintinline{c}{name[8:15]}}{7fb0}}
                    \node[draw,minimum height=1cm,anchor=east] (n8) at (2, \value{cellnb}) {\only<1>{\mintinline{c}{n[8]}}\only<2->{\mintinline{c}{'A'}}};
                    \node[] at (0, \value{cellnb}) {\dots};
                    \node[draw,minimum height=1cm,anchor=west] (n15) at (-2, \value{cellnb}) {\only<1>{\mintinline{c}{n[15]}}\only<2->{\mintinline{c}{'A'}}};
                    \cell{} \cellcom{\naddr{\mintinline{c}{name[0:7]}}{7fa8}}
                    \node[draw,minimum height=1cm,anchor=east] (n0) at (2, \value{cellnb}) {\only<1>{\mintinline{c}{n[0]}}\only<2->{\mintinline{c}{'A'}}};
                    \node[] at (0, \value{cellnb}) {\dots};
                    \node[draw,minimum height=1cm,anchor=west] (n7) at (-2, \value{cellnb}) {\only<1>{\mintinline{c}{n[7]}}\only<2->{\mintinline{c}{'A'}}};
                    \finishframe{{\tt callme}\\stack frame}
    
                    \draw[->] (2.15,0) -- node[auto, sloped]{stack growth} (2.15, \value{cellnb}-1);
                \end{drawstack}
                  
                
                \caption{The Stack: Smashed}
                \label{fig:stack2}
            \end{figure}
        \end{column} 
        \end{columns}
    
\end{frame}
\begin{frame}[fragile]
    \frametitle{Example 2.0: Smashing the Stack for Fun and Profit}
    \begin{itemize}
        \item What happens now?
    \end{itemize}
    \pause
    \begin{alertblock}{Segmentation Fault}
        \inlinecode[bash]{Program received signal SIGSEGV, Segmentation fault.}\\
        If we would look at the program in a debugger, we would see that it tried executing code at address \hex{4141414141414141}!
        Mission accomplished!
    \end{alertblock}

\end{frame}
\begin{frame}[fragile]
	\frametitle{Return Oriented Programming}
	\begin{itemize}
		\item While this is already cool, we need to have a useful location to jump to, otherwise this alone is useless
		\item Enter Return Oriented Programming (ROP)
		\item By finding small snippets of useful assembly code in the binary (or others loaded by it), that we can chain together (ROP chain), to achieve a goal
		\pause
		\vspace{2em}
		\begin{itemize}
			\item Gadgets are snippets of assembly that end with a "ret" instructions
			\item Since we assume we control the stack, we also control all the future "return addresses"
			\item So we can chain gadgets together, to form a \emph{shellcode}!
		\end{itemize}
	\end{itemize}
\end{frame}
\begin{frame}[fragile]
	\frametitle{Return Oriented Programming 2}
	\begin{itemize}
		\item Now ROPing is not exactly like shellcoding. 
		\item Example: to set register rbx, you would use the following asm snippet: \inlinecode[bash]{pop rbx; ret}
		\\ 
		\pause
		\item PAY ATTENTION TO THE CALLING CONVENTION!
		\vspace{2em}
		\begin{itemize}
			\item You want to call system("/bin/sh")
			\item on x86-64, you need to set registers rdi, rsi, and so on
			\item on x86-32, arguments are passed on the stack!
		\end{itemize}
	\end{itemize}
\end{frame}
\begin{frame}[fragile]
    \frametitle{Finding Gadgets and Addresses}
    \begin{itemize}
		\item Looking through the disasm different functions you can find gadgets by hand
        \item A better way is to use \inlinecode[bash]{ropper} to find gadgets
        \item Pay attention to values you want to have on the stack are not nice characters (for example \hex{80})
    \end{itemize}
\end{frame}
\section{Challenge}
\begin{frame}
    \frametitle{Challenge}
    {
        \begin{alertblock}{\textbf{ropperoni}}
            Same challenge as last week, but this time there is no win function!\\
			Canary was also disabled to avoid bruteforcing it every time. \\
            \textbf{Files:} on discord channel "fridays-meetings"\\
            \textbf{Server:} cyanpencil.xyz 1337\\
            \textbf{Author:} Florian Hofhammer (go complain to him if it is broken)
        \end{alertblock}
    }
\end{frame}
\section{Further Readings}
\begin{frame}[fragile]
    \frametitle{Assembly, Disassemblers and Decompilers}
    \begin{itemize}
        \item x86 Assembly
        \begin{itemize}
            \item "Machine Language" of modern Intel/AMD processors and so most binary challenges use it
            \item \href{http://www.cs.virginia.edu/~evans/cs216/guides/x86.html}{x86 Assembly Guide}
            \item \href{http://gec.di.uminho.pt/DISCIP/MaisAC/CS-APP_Bryant/csapp.preview3.pdf}{CS:APP Chapter 3: Machine-Level Representation of Programs}
        \end{itemize}
        \item Disassemblers
        \begin{itemize}
            \item Take a binary and display the x86 assembly nicely, cross reference stuff, etc.
            \item \inlinecode[bash]{objdump -D binary} is the simplest version
            \item \href{https://cutter.re}{Cutter (powered by radare2)}
        \end{itemize}
        \item Decompilers
        \begin{itemize}
            \item Take the output of a dissassembler and try to create C code that does the same thing as the assembly
            \item Usually Programms do both
            \item Ghidra is pretty good and available for free at \href{https://ghidra-sre.org}{ghidra-sre.org}
            \item \href{https://ghidra.re/courses/GhidraClass/Beginner/Introduction_to_Ghidra_Student_Guide.html}{Introduction to Ghidra}
        \end{itemize}
    \end{itemize}

\end{frame}
\begin{frame}
    \frametitle{Advanced ROP Techniques and Protection Measures}
    \begin{itemize}
        \item ROPing all the things!
        \begin{itemize}
            \item \href{https://www.sciencedaily.com/releases/2009/08/090810161902.htm}{ROP on a voting machine}
            \item \href{https://www.fireeye.com/blog/threat-research/2013/02/the-number-of-the-beast.html}{ROP on an Adobe Reader PDF}
        \end{itemize}
        \item Must-have tools for ROP:
        \begin{itemize}
            \item \href{http://shell-storm.org/project/ROPgadget/}{ROPgadget}
            \item \href{https://scoding.de/ropper/}{ropper}
        \end{itemize}
        \item Cool ROP Techniques:
        \begin{itemize}
            \item \href{https://en.wikipedia.org/wiki/Return-to-libc_attack}{(basic) ret2libc}
            \item \href{https://en.wikipedia.org/wiki/Sigreturn-oriented_programming}{(hard) sigreturn oriented programming}
        \end{itemize}
        \item Defending against ROPs:
        \begin{itemize}
            \item \href{https://en.wikipedia.org/wiki/Address_space_layout_randomization}{ASLR}
            \item \href{http://s3.eurecom.fr/docs/acsac10_gfree.pdf}{G-free}
            \item \href{https://www.qualcomm.com/media/documents/files/whitepaper-pointer-authentication-on-armv8-3.pdf}{PAC: Pointer Authenication} (use crypto to secure return pointers)
        \end{itemize}
    \end{itemize}
\end{frame}
\end{document}
