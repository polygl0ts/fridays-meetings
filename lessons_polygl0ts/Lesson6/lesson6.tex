\documentclass[aspectratio=169]{beamer}
\usepackage[normalem]{ulem}


\usepackage{../theme/polygl0ts}
\usepackage{pgfpages}

%\setbeameroption{show notes on second screen}

\tikzstyle{freecell}=[fill=none]

\newcommand{\reg}[1]{\%\mintinline{asm}{#1}}
\newcommand{\hex}[1]{\mintinline{python}{0x#1}}
\newcommand{\naddr}[2]{\begin{tabular}{l}#1\\\hex{#2}\end{tabular}}
\newcommand{\docl}[1]{(\textbf{\href{#1}{Documentation}})}

\hypersetup{colorlinks,linkcolor=,urlcolor=brightblue}

\setbeamertemplate{navigation symbols}{}

\showsectionframe

%%%%%%%%%%%%%%%%%%%%%%%%%%%%%%%%%%%%%%%%%%%%%%%%%%%%%%%%%%%%%%%%%%%%%%%%%%%%%%%
% Title Setup
%%%%%%%%%%%%%%%%%%%%%%%%%%%%%%%%%%%%%%%%%%%%%%%%%%%%%%%%%%%%%%%%%%%%%%%%%%%%%%%
\title{Lesson 6: Introduction to Reversing C++ Binaries}
\subtitle{Please Stop Compiling With O3}
\author{Leonardo Galli}
\date{\today}

\begin{document} 
\titleframe

\tocframe
\section{Readying IDA}
\begin{frame}
	\frametitle{Useful Plugins}
	\begin{itemize}
		\item \href{https://github.com/igogo-x86/HexRaysPyTools}{HexRaysPyTools}: Extremely useful for quickly creating structures without having to find every offset that might be a field.
		\item \href{https://github.com/RicBent/Classy}{Classy}: Makes working with vtables and child classes a lot easier.
	\end{itemize}
\end{frame}
\begin{frame}
	\frametitle{Other Settings}
	\begin{itemize}
		\item Make sure to regularly create a snapshot of your database! (File $\to$ Take Database Snapshot)
		\item Create / Open Classy Database (Classy $\to$ Create / Open)
		\item Make sure that compiler options are correct (Options $\to$ Compiler)
		\item (Optional) Always show demangled names (Options $\to$ Demangled names $\to$ Select Names)
	\end{itemize}
\end{frame}

\section{Theory}
\begin{frame}[fragile]
	\frametitle{C++ Class Layout in Memory}

\begin{codebox}{c}
// Usually stored in the data section.
struct vtable {
	void (*func1)();
	void (*func2)();
};

struct class {
	vtable* vtbl;
	int member1;
	int member2;
};
\end{codebox}

\end{frame}


\begin{frame}[fragile]
	\frametitle{C++ Patterns in a Decompiler}
\begin{codebox}{c}
// Initialize new instance of class
__int64 v1 = operator new(sizeof(class));
*v1 = gvtable; // stored somewhere in data.
*(v1 + 4) = 0;
*(v1 + 8) = 0;
\end{codebox}
\begin{codebox}{c}
// Call a vtable function
(*(void (*)())(*(_QWORD*)v1 + 8))();
\end{codebox}

\end{frame}

\section{Reconstructing Classes}

\begin{frame}[fragile]
    \frametitle{Finding vtables and Inheritence Hierarchy}
    \begin{itemize}
        \item Search for vtable in IDA, you should find something like this:
    \end{itemize}
\begin{codebox}{asm}
40FFC0 ; `vtable for'Polygon
40FFC0 _ZTV7Polygon    dq 0                    ; offset to this
40FFC8                 dq offset _ZTI7Polygon  ; `typeinfo for'Polygon
40FFD0 vtable      dq offset sub_408D40    ; DATA XREF: sub_408D40+11↑o
\end{codebox}
\begin{itemize}
    \item Go to \inlinecode[asm]{_ZTI7Polygon} and search cross references for ``reference to parent's type name'', giving you a location close to any child classes:
\end{itemize}
\begin{codebox}{asm}
3D40 ; public Triangle :
3D40 ;   public /* offset 0x0 */ Polygon
3D40 ; `typeinfo for'Triangle
\end{codebox}
\end{frame}

\begin{frame}
    \frametitle{Using Classy}
\begin{itemize}
    \item With the information gathered from before, start creating the hierarchy in classy
    \item Then add the correct vtable to every class in classy
    \item If possible, rename the functions correctly and add arguments in classy
    \item Details in demo later!
\end{itemize}
\end{frame}

\begin{frame}
    \frametitle{Creating Structures for Members}

\begin{itemize}
    \item Start with the base class and search for cross references to the vtable
    \item These are locations where class is constructed
    \item Use Structure Builder (right click $\to$ Show Structure Builder) to scan variable that is assigned the vtable
    \item This will automatically try to figure out how the struct layout should look like
    \item Go into any functions that use the newly allocated struct and scan as well
    \item Once satisfied, click finalize, you will be prompted to save the struct
    \item We first need to make some changes!
\end{itemize}
\end{frame}

\begin{frame}[fragile]
    \frametitle{Allowing Inheritence}
\begin{itemize}
    \item Surround everything except the first vtable field in another struct, named type\_members, e.g.:
\begin{codebox}{c}
struct class {
    vtable* vtbl;
    struct class_members {
        int member1;
        int member2;
    } mbrs;
};
\end{codebox}
    \item Now you can hit save
\end{itemize}
\end{frame}

\begin{frame}[fragile]
    \frametitle{Adding Subclasses}
\begin{itemize}
    \item Works very similar to the base class, but you search for cross references to the vtable of the subclass
    \item Also, you want to have the members struct inherit from the base member struct and delete any fields that are duplicate, e.g.:
\begin{codebox}{c}
struct subclass {
    subvtable* vtbl;
    struct subclass_members : class_members {
        // int member1; dup
        // int member2; dup
        int member3;
    } mbrs;
};
\end{codebox}
\end{itemize}
\end{frame}

\begin{frame}
    \frametitle{Renaming Vtable Functions}
    \begin{itemize}
        \item Once you created a struct, renaming vtable functions is not as easy anymore.
        \item If you rename them, the created vtable struct will not be automatically renamed as well!
        \item However, you can just go to the location of the vtable in the data section and press V
        \item This will ``recreate'' the vtable fixing up the namings!
    \end{itemize}
\end{frame}

{
\showsubsectionframe

\section{C++ STL}
\begin{frame}[fragile]
	\frametitle{What is the C++ STL?}
	\begin{itemize}
		\item Standard Template Library, is the library containing all the C++ types you know and love: \inlinecode[cpp]{std::string, std::vector, std::map}
		\item Two major issues present itself when reversing C++ binaries with STL types:
		\begin{itemize}
			\item When compiled with \inlinecode[c]{O3}, about 90\% of STL code will be inlined
			\item Memory layouts of STL types are different for every major OS and often not very intuitive
		\end{itemize}
	\end{itemize}
\end{frame}

\subsection{Strings}
\begin{frame}[fragile]
    \frametitle{Memory Layout}
    \begin{itemize}
        \item struct of size 32, if string is less than 16 bytes, everything is stored in the struct
        \item otherwise, allocated on heap, in steps of powers of 2
    \end{itemize}
\begin{codebox}{c}
struct basic_string
{
    char *begin_; // actual string data
    size_t size_; // actual size
    union
    {
        size_t capacity_; // used if larger than 15 bytes
        char sso_buffer[16]; // used if smaller than 16 bytes
    };
};	
\end{codebox}
\end{frame}

\begin{frame}[fragile]
    \frametitle{Inlined Initializers}
\begin{codebox}{c}
// v47 is of type basic_string!
v48 = 0LL;
v47 = (__int64)&v49;
LOBYTE(v49) = 0;
// once retyped:
v47.size_ = 0LL;
v47.begin_ = v47.sso_buffer;
v47.sso_buffer[0] = 0;
\end{codebox}

\end{frame}

\begin{frame}[fragile]
    \frametitle{Inlined Constructors}
\begin{codebox}{c}
// somewhere in the function, you might have this:
std::__throw_logic_error("basic_string::_M_construct null not valid");
// The whole function is probably just a string constructor,
// possible signatures:
string_construct(basic_string*, char* begin, char* end);
string_construct(basic_string*, basic_string*); // Copy
string_construct(basic_string*, char* begin, size_t size);
\end{codebox}
\end{frame}

\subsection{Vectors}
\begin{frame}[fragile]
    \frametitle{Memory Layout}
    \begin{itemize}
        \item struct of size 24, stores start, end and max pointer
        \item array is allocated on the heap, pointer type is dependent on vector elements
    \end{itemize}
\begin{codebox}{c}
// Stores Point objects
struct vector_point
{
    Point* start;
    Point* end;
    Point* max;
};
// Stores Point pointers, more common
struct vector_point_p
{
    Point** start;
    Point** end;
    Point** max;
};
\end{codebox}
\end{frame}

\begin{frame}[fragile]
    \frametitle{Inlined Size}
\begin{itemize}
    \item In case of storing pointers, size calculation is straight forward:
\end{itemize}
\begin{codebox}{c}
size_t size = (vec.end - vec.start) >> 3; // div by 8
\end{codebox}
\begin{itemize}
    \item In other cases, the division might look more painful:
\end{itemize}
\begin{codebox}{c}
// assume vec is a vector<char[5]>;
// div by 5
size_t size = ((vec.end - vec.start) * 0xCCCCCCCCCCCCCCCD) >> 2;
\end{codebox}
\end{frame}

\begin{frame}[fragile]
    \frametitle{Inlined Methods}
\begin{codebox}{c}
// Will often contain something like this:
std::__throw_out_of_range_fmt
// However, might not indicate that the whole function is from STL!
// Common append inlined method:
Point** end = vec->end;
if ( end == vec->max ) {
    // allocate more memory
    result = sub_4090C0(&vec->start, end, (Point *)&newp);
} else {
    if ( end ) {
        result = newp;
        *end = newp;
    }
    vec->end = end + 1;
}
\end{codebox}
    
\end{frame}

\subsection{Maps}

\begin{frame}[fragile]
    \frametitle{Memory Layout}
    \begin{itemize}
        \item Implemented using a red-black tree, so complex memory layout
    \end{itemize}
\begin{codebox}{c}
enum std::_Rb_tree_color : __int32
{
    _S_red = 0x0,
    _S_black = 0x1,
};
struct std::_Rb_tree_node_base
{
    std::_Rb_tree_color _M_color;
    struct std::_Rb_tree_node* _M_parent;
    struct std::_Rb_tree_node* _M_left;
    struct std::_Rb_tree_node* _M_right;
};
\end{codebox}
\end{frame}

\begin{frame}[fragile]
    \frametitle{Memory Layout}
\begin{codebox}{c}
struct std::map
{
    void* allocator; // uninteresting
    std::_Rb_tree_node_base _M_header;
    size_t _M_node_count;
};
// For a map of type map<string, Point*>
struct std::_Rb_tree_node : std::_Rb_tree_node_base
{
    struct string_point_pair
    {
        basic_string string;
        Point* point;
    } pair;
};
\end{codebox}
\end{frame}

\begin{frame}[fragile]
    \frametitle{Inlined Initalization}
\begin{codebox}{c}
v42 = operator new(0x30LL);
*(_DWORD *)(v42 + 8) = 0;
*(_QWORD *)(v42 + 16) = 0LL;
*(_QWORD *)(v42 + 40) = 0LL;
*(_QWORD *)(v42 + 24) = v42 + 8;
*(_QWORD *)(v42 + 32) = v42 + 8;
// After applying type:
map = (std::map*)operator new(0x30LL);
map->_M_t._M_impl._M_header._M_color = 0;
map->_M_t._M_impl._M_header._M_parent = 0LL;
map->_M_t._M_impl._M_node_count = 0LL;
map->_M_t._M_impl._M_header._M_left = &map->_M_t._M_impl._M_header;
map->_M_t._M_impl._M_header._M_right = &map->_M_t._M_impl._M_header;
\end{codebox}
    
\end{frame}
}

\section{Demo}
\begin{frame}
	\frametitle{Demo Time}
\end{frame}

\end{document}
